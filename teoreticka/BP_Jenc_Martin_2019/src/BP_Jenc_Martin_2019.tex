% arara: xelatex
% arara: xelatex
% arara: xelatex


% options:
% thesis=B bachelor's thesis
% thesis=M master's thesis
% czech thesis in Czech language
% english thesis in English language
% hidelinks remove colour boxes around hyperlinks

\documentclass[thesis=B,english]{FITthesis}[2012/10/20]

% \usepackage[utf8]{inputenc} % LaTeX source encoded as UTF-8
% \usepackage[latin2]{inputenc} % LaTeX source encoded as ISO-8859-2
% \usepackage[cp1250]{inputenc} % LaTeX source encoded as Windows-1250

\usepackage{graphicx} %graphics files inclusion
% \usepackage{subfig} %subfigures
% \usepackage{amsmath} %advanced maths
% \usepackage{amssymb} %additional math symbols

\usepackage{dirtree} %directory tree visualisation

% % list of acronyms
% \usepackage[acronym,nonumberlist,toc,numberedsection=autolabel]{glossaries}
% \iflanguage{czech}{\renewcommand*{\acronymname}{Seznam pou{\v z}it{\' y}ch zkratek}}{}
% \makeglossaries

% % % % % % % % % % % % % % % % % % % % % % % % % % % % % % 
% EDIT THIS
% % % % % % % % % % % % % % % % % % % % % % % % % % % % % % 

\department{Department of Applied Mathematics}
\title{Probabilistic algorithms for computing the LTS estimate}
\authorGN{Martin} %author's given name/names
\authorFN{Jen{\v c}} %author's surname
\author{Martin Jen{\v c}} %author's name without academic degrees
\authorWithDegrees{Martin Jen{\v c}} %author's name with academic degrees
\supervisor{Ing. Karel Klouda, Ph.D.}
\acknowledgements{THANKS to everybody}
\abstractEN{The least trimmed squares (LTS) method is a robust version of the classical method of least squares used to find an estimate of coefficients in the linear regression model. Computing the LTS estimate is known to be NP-hard, and hence suboptimal probabilistic algorithms are used in practice.}
\abstractCS{V n{\v e}kolika v{\v e}t{\' a}ch shr{\v n}te obsah a p{\v r}{\' i}nos t{\' e}to pr{\' a}ce v {\v c}esk{\' e}m jazyce.}
\placeForDeclarationOfAuthenticity{Prague}
\keywordsCS{LTS odhad, lineární regrese, optimalizace, nejmenší usekané čtvrece, metoda nejmenších čtverců, outliers}
\keywordsEN{LTS, linear regressin, robust estimator, least trimmed squares, ordinary least squares, outliers, outliers detection}
\declarationOfAuthenticityOption{1} %select as appropriate, according to the desired license (integer 1-6)
% \website{http://site.example/thesis} %optional thesis URL

%---------------------------------- main document -------------------------------------------------
% Kile, TeX Maker
%online - Pepeeria, >>Overleaf
\begin{document}



% \newacronym{CVUT}{{\v C}VUT}{{\v C}esk{\' e} vysok{\' e} u{\v c}en{\' i} technick{\' e} v Praze}
% \newacronym{FIT}{FIT}{Fakulta informa{\v c}n{\' i}ch technologi{\' i}}


% .......................... chapters .........................
%\setsecnumdepth{part}
%\include{00_intro}

% \setsecnumdepth{all}
% \include{01_opt}
% \include{02_ke}
% \include{03_alg}
% \include{04_apps}

%\setsecnumdepth{part}
%\include{99_concl}

% ***********************************************

\setsecnumdepth{part}
\chapter{Introduction}

\setsecnumdepth{all}
\chapter{The Least trimmed squares}
\subsection{Objective function}
\subsubsection{something}

\section{Robust statistics}
\section{Description}
\section{Computation}


\chapter{Algorithms}

% **********************************************************************************
% ************************* FAST - LTS *********************************************
% **********************************************************************************
\section{FAST-LTS}
In this section we'll describe FAST-LTS algorithm and it's main properties. The main idea of this algorthm is based on the fact that from one approximation of the algorithm we can compute another which can have lower objective function.
 TA DAAAAAAAAAAAAAAAAA
Thoerem 1: \cite{rybicka}
Let w0 ... wp be the LTS estimate.
for each data sample we can compute |y-wx|



Hlavni myslenka tohoto algoritmu spociva ve faktu, 


Vyberte si šablonu podle druhu práce (bakalářská, diplomová), jazyka (čeština, angličtina) a kódování (ASCII, \mbox{UTF-8}, \mbox{ISO-8859-2} neboli latin2 a nebo \mbox{Windows-1250}). 

V~české variantě naleznete šablony v~souborech pojmenovaných ve formátu práce\_kódování.tex. Typ práce může být:
\begin{description}
	\item[BP] bakalářská práce,
	\item[DP] diplomová (magisterská) práce.
\end{description}
Kódování zdrojového souboru (\LaTeX{}), ve kterém chcete psát, může být:
\begin{description}
	\item[UTF-8] kódování Unicode,
	\item[ISO-8859-2] latin2,
	\item[Windows-1250] znaková sada 1250 Windows.
\end{description}
V~případě nejistoty ohledně kódování doporučujeme následující postup:
\begin{enumerate}
	\item Otevřete šablony pro kódování UTF-8 v~editoru prostého textu, který chcete pro psaní práce použít -- pokud můžete texty s~diakritikou normálně přečíst, použijte tuto šablonu.
	\item V~opačném případě postupujte dále podle toho, jaký operační systém používáte:
	\begin{itemize}
		\item v~případě Windows použijte šablonu pro kódování \mbox{Windows-1250},
		\item jinak zkuste použít šablonu pro kódování \mbox{ISO-8859-2}.
	\end{itemize}
\end{enumerate}


V~anglické variantě jsou šablony pojmenované podle typu práce, možnosti jsou:
\begin{description}
	\item[bachelors] bakalářská práce,
	\item[masters] diplomová (magisterská) práce.
\end{description}

% **********************************************************************************
% ************************* EXACT POLYNOMIAL ALGORITHM *****************************
% **********************************************************************************
\section{Exact algorithm}
\section{Feasible solution}
\section{MMEA}
\section{Branch and bound}
\section{Adding row}


\chapter{Experiments}
\section{Data}
\section{Performance}
\section{Outlier detection}

\setsecnumdepth{part}
\chapter{Conclusion}

% citace - TODO
\bibliographystyle{iso690}
\bibliography{mybibliographyfile}

\setsecnumdepth{all}
\appendix

\chapter{Datasets}
% \printglossaries
\begin{description}
	\item[GUI] Graphical user interface
	\item[XML] Extensible markup language
\end{description}


\chapter{Contents of enclosed CD}

%change appropriately

\begin{figure}
	\dirtree{%
		.1 readme.txt\DTcomment{the file with CD contents description}.
		.1 exe\DTcomment{the directory with executables}.
		.1 src\DTcomment{the directory of source codes}.
		.2 wbdcm\DTcomment{implementation sources}.
		.2 thesis\DTcomment{the directory of \LaTeX{} source codes of the thesis}.
		.1 text\DTcomment{the thesis text directory}.
		.2 thesis.pdf\DTcomment{the thesis text in PDF format}.
		.2 thesis.ps\DTcomment{the thesis text in PS format}.
	}
\end{figure}

\end{document}
