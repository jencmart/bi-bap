\chapter{Algorithms}

In this chapter we'll introduce currently popular algorithms of computing FAST-LTS estimate.

% \textbf{X} - bold
% \pmb{X} - strong bolda
% \boldmath{X} - quite normal
%  $\textbf{X}^T$ - TRANSPOSED MATRIX

% change arrowed vector to the bold vector
% mathbf give bold 
% bold-symbol gives bold cursive
% **********************************************************************************
% *************% ************* ************ FAST - LTS *********************************************
% ************* ************ FAST - LTS *********************************************
% ************* ************ FAST - LTS *********************************************
% ************* ************ FAST - LTS *********************************************
% ************ FAST - LTS *********************************************
% **********************************************************************************
\section{FAST-LTS}
In this section we will introduce FAST-LTS algorithm\cite{rouss:2000}. 
It is, as well as in other cases, iterative algorithm. We will discuss all main components
of the algorithm starting with its core idea called concentration step which '
authors simply call C-step.

% $\showVec{\hat{w}}{w}{n}$
% $\boldsymbol{Y}$ - bold cursive

\subsection{C-step}
We will show that from existing LTS estimate $\boldsymbol{\hat{w}_{old}}$ we 
can construct new LTS estimate $\boldsymbol{\hat{w}_{new}}$ which objective 
function is less or equal to the old one. Based on this property we will be able 
to create sequence of LTS estimates which will lead to better results.

% \usepackage{etoolbox}


%******************************** C-STEP theorem  ***************************************************%

\begin{theorem}
Consider dataset consisting of
$\vec{x_1}, \vec{x_2} \ldots,\vec{x_n}$ explanatory variables where 
$\vec{x_i}\in\mathbb{R}^p\,, \forall \vec{x_i} = (fx^i_1, x^i_2,\ldots,x^i_p)$ where $x^i_1 = 1$
and its corresponding $y_1, y_2,\ldots,y_n$ response variables. 
Let's also have $\vec{\hat{w}_0}\in\mathbb{R}^p$ any p-dimensional vector and 
$H_0 = \{{h_i ; h_i \in\mathbb{Z}\,, 1 \leq h_i \leq n\}}\,, |H_0| = h$. 
Let's now mark $RSS(\what{0}) = \sum_{i\in H_0} (r_0(i))^2$ where 
$r_0(i) = y_i - (w_1^0x^i_1 + w_2^0x^i_2 +\ldots+ w_p^0x^i_p$).
%Let's now mark  $H_1 = \{{h_i ;  1 \leq h_i \leq n\   \}}$ 
%uch that 
Let's take $\hat{n} = \{{1,2,\ldots,n\}}$ and mark
$\pi: \hat{n} \rightarrow \hat{n}$ permutation of $\hat{n}$ such that $|r_0({\pi(1)})| \leq |r_0({\pi(2)})| \leq \ldots \leq |r_0({\pi(n)})|$
and mark $H_1 = \{{\pi(1)\,, \pi(2)\,,... \pi(h)\}}$ set of $h$ indexes corresponding to $h$ smallest absolute residuals $r_0(i)$.
Finally take $\vec{\hat{w}^{OLS(H_1)}_1 }$ ordinary least squares fit on $H_1$ subset of observations
and its corresponding $RSS(\what{1}) = \sum_{i\in H_1} (r_1(i))^2$ sum of least squares. Then
\[ 
	RSS(\what{1}) \leq RSS(\what{0}) \numberthis
\]
\end{theorem}

\begin{proof}
	Because we take $h$ observations with smallest absolute residuals $r_0$, then for sure $\sum_{i\in H_1} (r_0(i))^2 \leq \sum_{i\in H_0} (r_0(i))^2 =  RSS(\what{0})$.
	When we take into account that Ordinary least squares fit $OLS_{H_1}$ minimize objective function of 
	$H_1$ subset of observations, then for sure  $RSS(\what{1}) =  \sum_{i\in H_1} (r_1(i))^2 \leq \sum_{i\in H_1} (r_0(i))^2$.
	Together we get $$RSS(\what{1})=\sum_{i\in H_1}(r_1(i))^2\leq\sum_{i\in H_1}(r_0(i))^2\leq\sum_{i\in H_0}(r_0(i))^2=RSS(\what{0})$$
\end{proof}

%******************************** C-STEP algorithm ***************************************************************************%

\begin{corollary} 
	Based on previous theorem, using some $\vec{\hat{w}^{OLS(H_{old})}}$  on $H_{old}$ subset of observations we can
	construct $H_new$ subset with corresponding $\vec{\hat{w}^{OLS(H_{new})}}$ such that $RSS(\vec{\hat{w}^{OLS(H_{new})}}) \leq RSS(\vec{\hat{w}^{OLS(H_{old})}})$. 
	With this we can apply above theorem again on $\vec{\hat{w}^{OLS(H_{new})}}$ with $H_{new}$. This will lead to the iterative sequence of
	$RSS(\what{{old}}) \leq RSS(\what{{new}}) \leq \ldots$. One step of this process is described by following pseudocode. Note that for C-step we actually need only $\vec{\hat{w}}$ 
	 without need of passing  $H$.
\end{corollary}

\begin{algorithm}[H]
	\label{alg:Cstep}
    % \SetKwInOut{Input}{input}
    % \SetKwInOut{Output}{output}
    \KwIn{dataset consiting of $\boldsymbol{X} \in \mathbb{R}^{n \times p}$ and $\boldsymbol{y} \in \mathbb{R}^{n \times 1}$,  $\what{{old}} \in \mathbb{R}^{p \times 1}$}
    \KwOut{ $\what{{new}}$, $H_{new}$ }
	\caption{C-step}
	
	$R \gets \emptyset$\;
	\For{$i \gets 1$ \textbf{to} $n$}{  
		$R \gets R \cup \{ |y_i - \what{{old}} \vec{x_i}^T |\}$\;
	}
	$H_{new} \gets $ select set of $h$ smallest absolute residuals from $R$\;
	$\what{{new}} \gets OLS(H_{new})$\;
	\Return{ $\what{{new}}$\,, $H_{new}$ }\;
\end{algorithm}

%******************************** C-STEP alg. time complexity ******************************************%
\begin{observation} 
	Time complexity of algorithm C-step \ref{alg:Cstep} is the same as time complexity as OLS. Thus $O(p^2n)$
	$\boldsymbol{{TODO}}$
\end{observation} 

\begin{lemma}
	Time complexity of OLS  on $\m{X}^{n \times p}$ and $\m{Y}^{n \times 1}$ is $O(p^2n)$.
\end{lemma}

\begin{proof}
	Normal equation of OLS is $\vec{\hat{w}} = (\m{X^T}\m{X})^{-1}\m{X^T}\m{Y}$.
	Time complexity  of matrix multiplication $\m{A}^{m \times n}$ and  $\m{B}^{n \times p}$ is $\sim \mathcal{O}(mnp)$.
	Time complexity of matrix $\m{C}^{m \times m}$ is $\sim \mathcal{O}(m^3)$
	So we need to compute 
	$\m{A} = \m{X^T}\m{X} \sim \mathcal{O}(p^2n)$ and
	$\m{B} = \m{X^T}\m{Y} \sim \mathcal{O}(pn)$ and
	$\m{C} = \m{A}^{-1} \sim \mathcal{O}(p^3)$ and finally 
	$\m{C}\m{B} \sim \mathcal{O}(p^2)$. 
	That gives us $\mathcal{O}(p^2n + pn + p^3 + p^2)$. Because  $\mathcal{O}(p^2n)$ and 
	$\mathcal{O}(p^3)$ asymptotically dominates over $\mathcal{O}(p^2)$ and $\mathcal{O}(pn)$ we can
	write $\mathcal{O}(p^2n + p^3)$.

	$\boldsymbol{{TODO}}$ CO zo toho je vic? Neni casove narocnejsi vynasobeni $\m{X^T}\m{X}$ nez inverze, kdyz bereme v uvahu $n >> p$ ???
\end{proof}

\begin{proof}
	In C-step we must compute $n$ absolute residuals. Computation of one absolute residual consists of
	matrix multiplication of shapes $1 \times p$ and $p \times 1$ that gives us $\mathcal{O}(p)$. Rest is in constant time.
	So time of computation $n$ residuals is $\mathcal{O}(np)$.
	Next we must select set of $h$ smallest residuals which can be done in $\mathcal{O}(n)$ using modification 
	of algorithm QuickSelect. reference: $\boldsymbol{{TODO}}$
	Finally we must compute $\hat{w}$ OLS estimate on $h$ subset of data.
	Because $h$ is linearly dependent on $n$, we can say that it is $\mathcal{O}(p^2n + p^3)$ which 
	is asymptotically dominant against previous steps which are $\mathcal{O}(np + n)$.
\end{proof}

As we stated above, repeating algorithm C-step will lead to sequence of $\what{1}, \what{2} \ldots$ 
on subsets $H_1, H_2 \ldots$ with corresponding residual sum of squares
$RSS(\what{{1}}) \geq RSS(\what{{2}}) \geq \ldots$. One could ask if this sequence will converge, so that
$RSS(\what{{i}}) == RSS(\what{{i+1}})$. 
Answer to this question will be presented by the following theorem.


%******************************** C-STEP alg. will converge ********************************************%
\begin{theorem}
	Sequence of C-step will converge to $\what{{m}}$ after maximum of $m = {n \choose h}$
	so that $RSS(\what{{m}}) == RSS(\what{{n}})\,, \forall n\geq m$ where $n$ is number of data samples 
	and $h$ is size of subset $H_i$.
\end{theorem}

\begin{proof}
	Since  $RSS(\what{{i}})$ is non-negative and $RSS(\what{{i}}) \leq RSS(\what{{i+i}})$ the 
	sequence will converge. $\what{{i}}$  is computed out of subset 
	$H_i \subset \{{1,2,\ldots,n\}}$. When there is finite number of subsets of size $h$ out of $n$ samples, namely ${n \choose h}$, the sequence will converge at the latest after this number of steps.
\end{proof}

%******************************** ITERATE-C-STEP algorithm **********************************************%
Above theorem gives us clue to create algorithm described by following pseudocode.

\begin{algorithm}[H]
	\label{alg:RepeatCstep}
	\KwIn{dataset consiting of $\boldsymbol{X} \in \mathbb{R}^{n \times p}$ 
	and $\boldsymbol{y} \in \mathbb{R}^{n \times 1}$,  $\what{{old}} \in \mathbb{R}^{p \times 1}\,, H_0 $}
    \KwOut{ $\what{{final}}$, $H_{final}$ }
	\caption{Repeat-C-step}
	\SetKw{Break}{break}
	$\what{{new}} \gets \emptyset$\;
	$H_{new} \gets \emptyset$\;
	$RSS_{new} \gets \infty $\;

	\While{$True$}{
		$RSS_{old} \gets RSS(\what{{old}})$\;
		$\what{{new}}$\,, $H_{new} \gets \boldsymbol{X}\,, \boldsymbol{y}\,, \what{{old}}$\;
		$RSS_{new} \gets RSS(\what{{new}})$\;
		\If{$RSS_{old} == RSS_{new}$}{
			\Break
		  }
		$\what{{old}} \gets \what{{new}}$
	}

	\Return{ $\what{{new}}$, $H_{new}$ }\;
\end{algorithm}

It is important to note, that although maximum number of steps of this algorithm is ${n \choose h}$ in practice it is very low, most often under $20$ steps. $\boldsymbol{TODO}$ nejaky hezky grafik ktery to ukazuje....
That is not enough for the algorithm $Repeat-C-step$ to converge to global minimum, but it is necessary condition. That gives us an idea how to create the final algorithm. \cite{rouss:2000}

Choose a lot of initial subsets $H_1$ and on each of them apply algorithm Repeat-C-step. From all converged subsets with corresponding $\hat{w}$ estimates choose that which has lowest $RSS(\hat{w})$. 

Before we can construct final algorithm we must decide how to choose initial subset $H_1$ and how many of them mean ``\emph{a lot of}''. First let's focus on how to choose initial subset $H_1$.

% \renewcommand{\O}[1]{$\mathcal{O}(#1)$}

% ************************************************	INITIAL H_1 SUBSET *************************************%
\subsection{Choosing initial $H_1$subset}

It is important to note, that when we choose $H_1$ subset such that it contains outliers, then iteration of  C-steps
usually won't converge to good results, so we should focus on methods with non zero probability of selecting $H_1$ such that it won't contain outliers.
There are a lot of possibilities how to create initial $hH_1$ subset. Lets start with most trivial one.


% ************************************************** RANDOM SELECTION **************************************%
\subsubsection{Random selection}
Most basic way of creating $H_1$ subset is simply to choose random $H_1 \subset \{{1,2,\ldots , n\}}$. Following observation will show that it not the best way.

\begin{observation}
	With increasing number of data samples, thus with increasing $n$, the probability of choosing among $m$ random selections of $H_{1_1}, \ldots ,H_{1_m}$ the probability of selecting
	at least one $H_{1_i}$ such that its corresponding data samples does not contains outliers, goes to $0$.
\end{observation}

\begin{proof}
	Consider dataset of $n$ containing $\epsilon > 0$ relative amount of outliers. Let $h=(n+p+1)/2$ and $m$ is number of selections random $|H| = h$ subsets. Then
	\begin{align*}
		P(one~random~data~sample~not~outliers) &= (1-\epsilon) \\
		P(one~subset~without~outliers) &= (1-\epsilon)^h \\
		P(one~subset~with~at~least~one~outlier) &= 1-(1-\epsilon)^h \\
		P(m~subsets~with~at~least~one~outlier~in~each) &= (1-(1-\epsilon)^h)^m \\
		P(m~subsets~with~at~least~one~subset~without~outliers) &= 1-(1-(1-\epsilon)^h)^m \\
	\end{align*}

	Because $n \rightarrow \infty 	
	\Rightarrow (1-\epsilon)^h  \rightarrow 0 	
	\Rightarrow 1- (1-\epsilon)^h  \rightarrow 1
	\Rightarrow (1-(1-\epsilon)^h)^m  \rightarrow 1
	\Rightarrow 1- (1-(1-\epsilon)^h)^m  \rightarrow 0 $
\end{proof}

That means that we should consider other options of selecting $H_1$ subset. Actually if we would like to continue with selecting some random subsets, previous observation gives us clue, that we should choose it independent of $n$. Authors of algorithm came with such solution and it goes as follows.

% ***************************************************** P - SELECTION ***************************************%
\subsubsection{P-subset selection}
Let's choose subset $J \subset \{{1,2,\ldots,n\}}\,, |J| = p$. Next compute rank of matrix $\m{X}_{J:}$. If $rank(\m{X}_{J:}) < p$ add randomly selected rows to $\m{X}_{J:}$ without repetition until $rank(\m{X}_{J:}) = p$. Let's from now on suppose that $rank(\m{X}_{J:}) = p$. Next let us mark $\what{0} = OLS(J)$ and corresponding $(r_0(1)), (r_0(2)), \ldots ,(r_0(n))$ residuals.  Now mark $\hat{n} = \{{1,2,\ldots,n\}}$ and let
$\pi: \hat{n} \rightarrow \hat{n}$ be permutation of $\hat{n}$ such that $|r({\pi(1)})| \leq |r({\pi(2)})| \leq \ldots \leq |r({\pi(n)})|$. Finally put $H_1 = \{{\pi(1)\,, \pi(2)\,,... \pi(h)\}}$ set of $h$ indexes corresponding to $h$ smallest absolute residuals $r_0(i)$.

\begin{observation}
	\label{cStepM}
	With increasing number of data samples, thus with increasing $n$, the probability of choosing among $m$ random selections of $J_{1_1}, \ldots ,J_{1_m}$ the probability of selecting
	at least one $J_{1_i}$ such that its corresponding data samples does not contains outliers, goes toho
	$$ 1-(1-(1-\epsilon)^h)^m  > 0$$
\end{observation}

\begin{proof}
	Similarly as in previous observation.
\end{proof}

\begin{itshape}
Note that there are other possibilities of choosing $H_1$ subset other than these presented in \cite{rouss:2000}.
We'll properly discuss them in chapter $\boldsymbol{TODO}$.\\
\end{itshape}

Last missing piece of the algorithm is determining number of $m$ initial $H_1$ subsets, which will maximize probability to at least one of them will converge to good solution. Simply put, the more the better. So before we will answer this question properly, let's discuss some key observations about algorithm.

% ***************************************************** SPEED-UP ***************************************%

\subsection{Speed-up of the algorithm}
In this section we will describe important observations which will help us to formulate final algorithm. In two subsections we'll briefly describe how to optimize current algorithm. 

% *************************************************SELECTIVE ITERATION  ***************************************%
\subsubsection{Selective iteration}
The most computationally demanding part of one C-step is computation of OLS on $H_i$ subset and then 
calculation of $n$ absolute residuals. How we stated above, convergence is usually achieved under 20 steps. 
So for fast algorithm run we would like to repeat C-step as little as possible and in the same time didn't loose performance of algorithm. 

Due to that convergence of repeating C-step is very fast, it turns out, that we are able to distinguish between starts that will lead to good solutions and those who won't even after very little C-steps iterations. <based on empiric observation, we can distinguish good or bad solution already after two or three iterations of C-steps based on $RSS(\what{3})$ or $RSS(\what{4})$ respectively. 

So even though authors don't specify size of $m$ explicitly, they propose that after a few C-steps we can choose (say~10) best solutions among all $H_1$ starts and continue C-steps till convergence only on those best solutions.
This process is called Selective iteration.

\begin{itshape}
	We can choose $m$ with respect to observation \ref{cStepM}. In ideal case we would like to have probability of existence at least one initial $H_1$ subset close to $1$. As we see $m$ is exponentially dependent on $p$ and at the same time in practice we don't know percentage of outliers in dataset. So it is difficult to mention exact value. Specific values of $m$ in respect to data size is visible in table $\boldsymbol{TODO}$. So we can say that with $p < 10$ choosing $m = 500$ is usually safe starting point.
\end{itshape}

\subsubsection{Nested extension}
C-step computation is usually very fast for small $n$. Problem starts with very high $n$ say $n > 10^3$ because we need to compute OLS on $H_i$ subset of size $h$ which is dependent on $n$. And then calculate $n$ absolute residuals.

Authors came up with solution they call Nested extension. We will describe it briefly now.
\begin{itemize}
	\item If $n$ is greater than limit $l$, we'll create subset of data samples $L\,, |L| = l$ and divide this subset into $s$ disjunctive sets $P_1,P_2,\ldots,P_s\,, |P_i| = \frac{l}{s}\,, P_i\cap P_j  = \emptyset\,, \bigcup_{i=1}^{s} P_{i} = L$.
	\item For every $P_i$ we'll set number of starts $m_{P_i} = \frac{m}{l}$. 
	\item Next in every $P_i$ we'll create $m_{P_i}$ number of initial $H_{P_{i_1}}$ subsets and iterate C-steps for two iterations.
	\item Then we'll choose $10$ best results from each subsets and merge them together. We'll get family of sets
	$F_{merged}$ containing $10$ best $H_{P_{i_3}}$ subsets from each $P_i$.
	\item On each subset from  $F_{merged}$ family of subsets we'll again iterate $2$ C-steps and then choose $10$ best results.
	\item Finally we'll use these best $10$ subsets and use them to iterate C-steps till convergence.
	\item As a result we'll choose best of those $10$ converged results.
\end{itemize} 

\subsubsection{Putting all together}
We've described all major parts of the algorithm FAST-LTS. One last thing we need to mention is that even though C-steps iteration usually converge under $20$ steps it is appropriate to introduce two parameters $max_iteration$ and $threshold$ which will limit number of C-steps iterations in some rare cases when convergence is too slow. Parameter $max_iteration$ denotes maximum number of iterations in final C-step iteration till convergence. Parameter $threshold$ denotes stopping criterion such that $| RSS(\what{{i}}) - RSS(\what{{i+1}})| \leq threshold$ instead of 
$RSS_{i} == RSS_{i+1}$ . When we put all together, we'll get FAST-LTS algorithm which is described by following pseudocode.


\begin{algorithm}[H]
	\label{alg:FAST-LTS}
	\KwIn{$\boldsymbol{X} \in \in \mathbb{R}^{n \times p}, \boldsymbol{y} \in \mathbb{R}^{n \times 1}, m, l, s, max_iteration, threshold $}
    \KwOut{ $\vec{\hat{w_{final}}}$, $H_{final}$ }
	\caption{FAST-LTS}
	\SetKw{Break}{break}
	$\what{{final}} \gets \emptyset$\;
	$H_{final} \gets \emptyset$\;
	$F_{best} \gets \emptyset$\;

	\uIf{$n \geq l$}{
		$F_{merged} \gets \emptyset$\;
		% for each split
		\For{$i \gets 0$ \textbf{to} $s$}{
			% create xx number of starts
			$F_{selected}  \gets \emptyset$\;
			\For{$j \gets 0$ \textbf{to} $\frac{l}{s}$}{
			  $F_{initial} \gets Selective~iteration(\frac{m}{l})$\;
			  % on each starts iterate few c steps
			  \For{$H_i$ \textbf{in} $F_{initial}$}{
				$H_i \gets Iterate~C~step~few~times(H_i)$\;
				$F_{selected} \gets  F_{selected} \cup \{{ H_i \}} $\;
			  }
			}
			% among all starts select 10 best and add it to merged set
			$F_{merged} \gets F_{merged} \cup Select~10~best~subsets~from~F_{selected}$\;
		  }
		
		  % for each, say 50 best, iterate few and add it to best
		  \For{$H_i$ \textbf{in} $F_{merged}$}{
			$H_i \gets Iterate~C~step~few~times(H_i)$\;
			$F_{best} \gets F_{best} \cup \{{ H_i \}} $\;
		  }
		  $F_{best} \gets  Select~10~best~subsets~from~F_{best} $\;
	}
	\Else{
		$F_{initial} \gets Selective~iteration(m)$\;
		$F_{best} \gets  Select~10~best~subsets~from~F_{initial} $\;
	}

	% iterate till convergence on few best final results
	$F_{final} \gets \emptyset$\;
	$W_{final} \gets \emptyset$\;
	\For{$H_i$ \textbf{in} $F_{best}$}{
			$H_i, \what{i} \gets Iterate~C~step~till~convergence(H_i, max_iteration, threshold)$\;
			$F_{final} \gets F_{final} \cup \{{ H_i \}}$\;
			$W_{final} \gets W_{final} \cup \{{ \what{i} \}}$\;
	}

	% select one best result
	$\what{{final}}, H_{final} \gets select~what~with~best~RSS(F_{final}, W_{final})$\;

	\Return{ $\what{{final}}, H_{final}$  }\;
\end{algorithm}


%********************************************************************************************************%
%******************************************** END *******************************************************%
%********************************************************************************************************%


% \begin{description}
% 	\item[BP] 
% 	\item[DP] 
% \end{description}


% **********************************************************************************
% ************************* EXACT POLYNOMIAL ALGORITHM *****************************
% **********************************************************************************
\section{Exact algorithm}
\section{Feasible solution}
\section{MMEA}
\section{Branch and bound}
\section{Adding row}
