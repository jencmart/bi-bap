
\chapter{Conclusion}
% Goal of this thesis was to compare probabilistic algorithms for computing the LTS estimate. 
We have surveyed, implemented and described multiple exact and probabilistic algorithms for calculating the LTS estimate.
Those algorithms have been proposed across the last few decades. Most recently proposed algorithms are just a few years old. This means that research in the filed of LTS algorithms is still ongoing.

Although the exact algorithms have polynomial time complexity, we showed that currently used probabilistic algorithm provide sufficiently fast solutions which, even though that may not be exact, are good enough. Despite the fact that it was proven that the exact solution cannot be obtained faster than in polynomial time, we showed that currently used algorithms could be combined to obtain better results. 


Algorithms for calculating the LTS estimate are still open topic for further research. One of the possible research direction is to study possibility to combine the exact algorithms with probabilistic ones. As our experimental results suggest, we could possibly come up probabilistic algorithms which provide even better performance.

Another direction could be to use algorithms for computing LTS estimate on different robust statistic methods because some of them are very similar to the LTS problem.
