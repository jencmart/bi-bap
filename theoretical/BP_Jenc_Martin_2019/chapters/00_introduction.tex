\chapter{Introduction}
Least trimmed squares (LTS) is one of the many modifications of the very well known method of ordinary least squares (OLS). Both these methods are tools of regression analysis, which is a group of the processes used to estimate the dependence of variables. Specifically in the case when we try to estimate dependence one variable on one or multiple others. The regression analysis uses the regression models, and in the case of OLS and LTS methods, that model is the linear regression model.

In order to OLS estimate produces reliable results, many strong assumptions about the data have to be fulfilled. It is assumed that data is generated in a specific way as well as that the data does not contain measurement errors called outliers. Those assumptions are in practice hardly fulfilled, because outliers in the data are very common. The OLS estimate is in such cases unreliable.

Robust statistic tries to solve problems of classical statistics methods. Its methods are usually emulations of classical statistic methods but try to provide reliable estimates even if data contains a large number of outliers. That means such methods does not rely so much on assumptions which are difficult to achieve in practice. Because the OLS method is one of the essential tools of linear regression analysis, multiple its alternatives have been designed to fulfill the assumptions of a robust estimator.

The idea of the LTS method is simple, but unlike the OLS method, the exact solution is known to be NP-hard, hence only suboptimal probabilistic algorithms are usually used in practice.

This work is divided into three chapters. In the first chapter, we introduce theory required to understand linear regression model, OLS, and LTS. We mention the properties of both methods and also describe the field of its usage.

In the second chapter, we cover algorithms for calculating the OLS estimate. It is necessary because most of the algorithm used to compute the LTS estimate relies on those algorithms. Next, we describe all currently used algorithms for calculating LTS estimate. They consist of multiple probabilistic and also few exact ones. In this chapter, we also show that those algorithms can be easily combined to obtain higher speed and performance. Last but not least, we also propose several improvements to those algorithms.

In the last chapter, we describe our experimental results. At first, we cover data generator which is used for our experiments and which can provide data sets affected by various types of outliers. Next, we provide information about our implementation of all algorithms from chapter two. Finally, we present our results for specific data sets.